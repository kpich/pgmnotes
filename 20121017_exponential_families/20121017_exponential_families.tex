\documentclass{article}

\usepackage{algpseudocode}
\usepackage{algorithm}
\usepackage{amsmath,amssymb}

\title{PGM Class Notes, Exponential Families}
\author{Karl Pichotta \\ pichotta@cs.utexas.edu}
\date{October 17, 2012}


\newcommand{\B}{\mathcal{B}}
\newcommand{\F}{\mathcal{F}}
\newcommand{\E}{\mathbb{E}}
\newcommand{\R}{\mathbb{R}}
\newcommand{\Z}{\mathbb{Z}}
\newcommand{\inv}{^{-1}}


\begin{document}

\maketitle




\section{Exponential Distributions}

$$
P(x;\eta) = h(x) \exp\{ \eta^TT(x) - A(\eta) \}
$$

We have
$$
\exp(A(\eta))
=
\int
h(x) 
\exp(\eta^TT(x))
\, dx
$$
and
$$
p(x;\eta)
=
\frac{h(x) \exp\{ \eta^T T(x)\}}
{\exp(A(\eta))}
$$


\section{Examples}

\begin{enumerate}
	\item
	\textbf{Bernoulli Dist.}
	Parametrized by $\pi$, bias of a coin. $X\in\{0,1\}$.
	\begin{align}
	p(x;\pi) &= \pi^x (1-\pi)^{1-x} \\
	&= \exp ( x \log \pi + (1-x) \log (1-\pi)) 
	\\
	&= \exp  \left( x \log \frac{\pi}{1-\pi} + \log (1-\pi) \right) 
	\\
	\end{align}
	So to make it exponential, we have
	$$
	\eta = -\log \frac{\pi}{1 - \pi}
	\Rightarrow
	\exp\eta = \pi / (1 - \pi)
	\Rightarrow
	1 - \pi = (1 = \exp\eta)\inv
	$$
	(might be missing a -1 sign)
	$$
	T(x) = x
	$$
	\begin{align*}
	A(\eta) &= - \log(1 - \pi)
	\\
	&= -\log((1 + \exp \eta)\inv)
	\\
	&= \log(1 + \exp \eta)
	\end{align*}
	
	
	
	\item
	\textbf{Gaussian}
	
	
	\begin{align}
	p(x;\mu,\sigma^2) &= 
	\frac{1}{\sigma\sqrt{2\pi}} \exp\left\{  \frac{1(x-\mu)^2}{2\sigma^2}   \right\}
	\\
	&= 
	\frac{1}{\sigma\sqrt{2\pi}} \exp\left\{  
		\frac{-x^2}{2\sigma^2}
		-
		\frac{\mu^2}{2\sigma^2}
		+
		\frac{x\mu}{2\sigma^2}
		-\log\sigma
	\right\}
	\\
	&= 
	\frac{1}{\sigma\sqrt{2\pi}} \exp\left\{  
		x^2\frac{-1}{2\sigma^2}
		+
		x
		\frac{\mu}{2\sigma^2}
		-
		\left(
		\frac{\mu^2}{2\sigma^2}
		+\log\sigma
		\right)
	\right\}
	\end{align}
	This is the ``moment representation''.
	
	So we have
	$$
	\eta = \langle
	\mu/\sigma^2
	\qquad
	-1/2\sigma^2
	 \rangle
	$$
	and
	$$
	T(x) = \langle   x\qquad x^2   \rangle
	$$
	
	$$
	A(\eta) = 
	\frac{-\eta_1^2}
	{4\eta_2}
	-
	1/2 \log(-2\eta_2)
	$$
	(I might have copied that wrong; double check it.)
	And so we get our canonical representation
	$$
	p(x;\eta) = (2\pi)^{-1/2} \exp\left\{   \eta_1 x + \eta_2 x^2  - A(\eta)  \right\}
	$$
	
	\item
	\textbf{Multinomial}
	$x = (x_1,\ldots,x_p)$
	parametrized by $\pi_1, \ldots, \pi_p$ s.t.\ $\pi_i\geq 0$ and $\sum \pi_i = 1$.
	Toss a die $n$ times, let $X_i$ be the number of times the die lands on face $i$.
	$$
	p(x;\pi) = 
	\frac{n!}
	{x_1!x_2!\ldots x_p!}
	\pi_1^{x_1}
	\cdots
	\pi_p^{x_p}
	$$
	So we can write this as
	\begin{align}
	p(x;\pi)
	&=
	\frac{n!}  {\prod_i x_i!}
	\exp
	\left\{
		x_1\log\pi_1 + \cdots + 
		x_p \log \pi_p
	\right\}
	\end{align}
	So we might think we can write
	$T(x) = \langle x_1, \ldots, x_p\rangle$
	and
	$\eta = \langle \log \pi_1 ,\ldots, \log\pi_p\rangle$
	This doesn't work though because there's a linear dependence between the $x$'s that we don't
	have here: $\sum_i x_i = n$.
	There's of course also one in the $\pi_i$'s.
	So we can express $x_p$ in terms of $x_1, \ldots, x_{1-p}$, ditto with the $\pi$'s.
	So we werite
	$$
	X_p = n - \sum_{i=1}^{p-1}X_i
	$$
	and
	$$
	\pi_p = 1 - \sum_{i=1}^{p-1}\pi_i
	$$
	So our joint becomes
	\begin{align}
	p(x;\pi)
	&=
	\frac{n!}  {\prod_i x_i!}
	\exp
	\left\{
		x_1\log\pi_1 + \cdots +
		\left(  n - \sum_{i=1}^{p-1}X_i \right)
		\log \left( 1 - \sum_{i=1}^{p-1}\pi_i \right)
	\right\}
	\\
	&=
	\frac{n!}  {\prod_i x_i!}
	\exp
	\left\{
		\sum_{i = 1}^{p-1}
		x_i\log \frac{\pi_i}{ 1 - \sum_{i\neq p} \pi_i}
		+
		n \log\left( 
		1 - \sum{i\neq p} \pi_i
		 \right)
	\right\}
	\end{align}
	giving us
	$$
	T(x) = \langle x_1, \ldots, x_p\rangle
	$$
	and
	$$
	\eta = 
	\left\langle
	\log \frac{\pi_1} {1 - \sum_{i\neq p} \pi_i}
	,
	\ldots
	,
	\log \frac{\pi_p} {1 - \sum_{i\neq p} \pi_i}
	\right\rangle
	$$
	with our $A$ term:
	$$
	A(\eta) =
	- n\log
	\left(
	1 - \sum_{i\neq p} \pi_i
	\right)
	$$
	Consider $\eta_i$:
	\begin{align}
	&
	\eta_i
	=
	\log
	\frac{\pi_i} {1 - \sum_{i\neq p} \pi_i}
	\\
	\Leftrightarrow \quad&
	\exp \eta_i
	=
	\frac{\pi_i} {1 - \sum_{i\neq p} \pi_i}
	\\
	\Leftrightarrow \quad&
	\sum_{i\neq p}
	\exp \eta_i
	=
	\frac{\sum_{i\neq p}\pi_i} {1 - \sum_{i\neq p} \pi_i}
	\\
	\Leftrightarrow \quad&
	\sum_{i\neq p}
	\pi_i
	=
	\frac{\sum_{i\neq p}\exp \eta_i} {1 - \sum_{i\neq p} \exp\eta_i}
	\\
	\Leftrightarrow \quad&
	1 - 
	\sum_{i\neq p}
	\pi_i
	=
	\frac{1} {1 + \sum_{i\neq p} \exp\eta_i}
	\end{align}
	(I don't get step (14), jeeez)
	and so
	$$
	A(\eta)
	=
	n \log \left(
	1 + \sum_{i\neq p} \exp(\eta_i)
	\right)
	$$
	Note that I think there's a minus sign somewhere that's flipped...
	
	
	\item
	\textbf{Poisson}
	The pmf is
	\begin{align}
	p(x;\lambda)
	&=
	\frac{e^{-\lambda} \lambda^x}{ x!}
	\\
	&=
	\frac{1}{x!} \exp\{
	x\log \lambda - \lambda
	\}
	\end{align}
	givig us
	$T(x) = x$, $\eta = \log \lambda$, and $A(\eta) = E^\eta$.
	And I can breathe a sigh of relief.
	
	
	
\end{enumerate}




\section{Moments and Canonical Parameters}

What's the expectation of an exponential family?
\begin{align*}
\E[X] = 
\int_{\mathcal X}
h(x) \exp\{
\eta^TT(x) - A(\eta)
\}
\, dx
\end{align*}
What is $A(\eta)$?
\begin{align*}
A(\eta) =
\log
\int
h(x) \exp\{
	\eta^T T(x)
\}
\, dx
\end{align*}
The grad is:
\begin{align*}
\nabla
A(\eta) =
\frac
{
\int T(x) h(x) \exp\{\eta^T T(x)\} \, dx
}
{
\int
h(x) \exp\{
	\eta^T T(x)
\}
\, dx
}
\end{align*}
The denominator is the normalization constant, So we have
\begin{align*}
\nabla
A(\eta) &=
\int
T(x) p(x;\eta)
\, dx
\\
&=
\E[T(x)]
= \mu
\end{align*}
So if our sufficient statistics are jsut $X$, then we can take the gradient of our log partition to get the first moment.
Hrm!

Note $\mu$ is a function of the parameters:
$$
\mu = \nabla A(\eta)
$$
if this is strictly convex, it is invertible:
$$
\Rightarrow
\eta = (\nabla A)\inv (\mu)
$$
So this lets us go from the first moment to the canonical parameters.

What's the second moment like? 
Recall
$$
A(\eta)
=
\log
\int
h(x) \exp(\eta^T T(X))
\,
dx
$$
So we have
\begin{align*}
A'(\eta)
&=
\frac
{\int T(x) h(x) \exp\{ \eta^T T(x)\}
}
{\exp(A(\eta))}
\end{align*}
Differentiating this again gives
\begin{align*}
A''(\eta)
&=
\int
T(x)
h(x)
\exp
\left\{
\eta^T T(x) - A(\eta)
\right\}
(T(x) - A'(\eta))
\,
dx
\\
&=
\int
T^2(x)
p(x;\eta)
\,
dx
-
A'(\eta)
\int
T(x) p(x;\eta)\, dx
\\
&=
\E[
T^2(x)
]
- \E[X]^2
\\
&=
Var(T(X))
\end{align*}
Rad. 



\subsection{Examples}

For the Gaussian recall we had
$$
	A(\eta) = 
	\frac{-\eta_1^2}
	{4\eta_2}
	-
	1/2 \log(-2\eta_2)
$$
with 
$$
	\eta = \langle
	\mu/\sigma^2
	\qquad
	-1/2\sigma^2
	 \rangle
$$
Take the grad wrt one eta at a time:
\begin{align*}
\frac{\partial A(\eta)}
{\partial \eta_1}
&=
\frac{-2\eta_1}
{4\eta_2}
\\
&=
\frac{\eta_1}
{2\eta_2}
\\
&=
\mu
\\
&=
\E[X]
\end{align*}
and
\begin{align*}
\frac{\partial A(\eta)}
{\partial \eta_2}
&=
\frac{\eta_1^2}
{4\eta_2^2}
-
\frac{2}
{4\eta_2^2}
\\
&=
\sigma^2 + \mu^2
\\
&=
\E[X^2]
\\
&=
Var(X)
+
\E[X]^2
\end{align*}
The second derivative gives us the second centered moment:
\begin{align*}
\frac{\partial^2 A(\eta)}
{\partial \eta_2^2}
&=
\frac{1}
{2\eta_2^2}
\\
&=
\sigma^2
\end{align*}

So $\mu$ and $\eta$ have a one-one mapping, so we can go from one to the other.
So if we have
$$
p(x;\eta)
=
\exp\{
\eta^T T(x) - A(\eta)
\}
$$
we can express this as a function of $\mu$, since $\eta$ is just a function of $\mu$:
$$
p(x;\mu)
=
\exp\{
\psi(\mu)^T T(x) - A(\psi(\mu))
\}
$$







\end{document}

































