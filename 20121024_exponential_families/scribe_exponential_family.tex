\documentclass[11pt]{article}

\usepackage{amsmath,amssymb}
\usepackage{amsthm}
\usepackage{bbm}
\usepackage{enumerate}
\usepackage[retainorgcmds]{IEEEtrantools}

\newcommand{\Bc}{\mathcal{B}}
\newcommand{\Fc}{\mathcal{F}}
\newcommand{\Xc}{\mathcal{X}}
\newcommand{\Vc}{\mathcal{V}}
\newcommand{\Ec}{\mathcal{E}}
\newcommand{\Sc}{\mathcal{S}}
\newcommand{\Mc}{\mathcal{M}}
\newcommand{\E}{\mathbb{E}}
\newcommand{\R}{\mathbb{R}}
\newcommand{\Z}{\mathbb{Z}}
\newcommand{\I}{\mathbb{I}}
\newcommand{\prob}{\mathbb{P}}
\newcommand{\1}{\mathbbm{1}}

\newtheorem{prop}{Proposition}
\newtheorem{thm}{Theorem}
\newtheorem{define}{Definition}
\newtheorem{cor}{Corollary}
\newtheorem{lem}{Lemma}

\title{PGM Scribe Note: Exponential Family 3}
\author{Xiaolong Li}
\date{\today}                                          

\begin{document}
\maketitle

\section{Mean Parameter}
Consider an exponential family parameterized by $\theta$:
\[
p(x, \theta) = \exp\{\langle \theta, \phi(x) \rangle - A(\theta)\}.
\]
Define \emph{mean parameter} as the expectation of the sufficient statistics,
$\mu_\alpha = \E_X[\phi_\alpha(x)]$, where $\alpha \in I$ and $|I| = d$ is the index set. More
over, we can define the mean parameter space as the set of all realizable mean
parameters
\[
\Mc = \{\mu \in \R^d \; | \; \mu = \E_\prob[\phi(x)] \text{ for some distribution } \prob \text{ over } x\}.
\]

\begin{prop}
$\Mc$ is a convex set.
\end{prop}
\begin{proof}
Let $\mu_1 = \E_{\prob_1}[\phi(x)]$, $\mu_2 = \E_{\prob_2}[\phi(x)]$.
Since for any $0 \leq \lambda \leq 1$, the convex combination $\lambda \prob_1 + (1-\lambda)\prob_2$ of two 
probability measure $\prob_1$ and $\prob_2$ is still a valid probability measure, and by linearility of measure
\[
\lambda \mu_1 + (1-\lambda)\mu_2 = \lambda \E_{\prob_1}[\phi(x)] + (1-\lambda) \E_{\prob_2}[\phi(x)]
= \E_{\lambda \prob_1 + (1-\lambda)\prob_2}[\phi(x)]
\]
This proves the claim.
\end{proof}

\begin{prop}
Suppose $X = (X_1, \ldots, X_p)$, $X_i \in \Xc$ where $\Xc$ is a finite set, then $\Mc$ is a polytope.
\end{prop}
\begin{proof}
$\mu \in \Mc$ if and only if $\mu = \sum_{x \in \Xc^p} p(x)\phi(x)$, where $p(x)$ is distribution over
finite set $\Xc^p$. Thus, by definition, $\Mc$ is the convex hull of finite set
$\{\phi(x_1), \ldots, \phi(x_n)\}$, $n = |\Xc|^p$. Since the convex hull of finite many points is always
a polytope, we have proved the claim.
\end{proof}

As a remark, there are two ways to characterize a polytope. One way, called V-representation, is to characterize
it as the convex hull of finite many points; another way, called H-representation, is to characterize it as
the intersection of finite many closed halfspaces. Note that for H-representation, a polytope might not be
bounded. Therefore, if we want to make it equivalent to V-representation, an additional boundedness condition
is necessary.

Consider $|\Xc| = 2$, then $\Mc$ has up to $2^p$ vertices and it is forbidding to use V-representation. However,
it might require far less number of halfsplaces if we use H-representation. We know that if $\Mc$ can be
represented efficiently, the inference will also be efficient.

(Potential project topic: In general, $r$ half spaces could create exponential vertices. But what about the
special case for tree? Is it efficiently V-representable?)

\section{Examples}
\subsection{Multi-variate Gaussian}
The sufficient statistics is $X$ and $XX^T$, then
\[
\Mc = \{(\mu, \Sigma) \; | \; \mu = \E_\prob[X], \Sigma = \E_\prob[XX^T] 
\text{ for some distribution } \prob \text{ over } x \}.
\]
$(\mu, \Sigma)$ is a valid parameter for some Gaussian distribution if and only if
$\Sigma \succeq \mu\mu^T$, which characterizes the shape of $\Mc$.
\footnote{We write $A \succeq B$ if $A-B$ is positive semi-definite.}

\subsection{Ising Model}
Suppose $X = (X_1, X_2, \ldots, X_p)$, $X_s \in \{0, 1\}$.\footnote{In last class, we said $X_s \in \{-1, 1\}$,
but its all the same after linear transformation.} The sufficient statistics is $X_s$ and $X_sX_t$ for
$(s,t) \in E$, so the mean parameter is
\[
\mu_s = \E[X_s] = \prob(X_s = 1) \in [0, 1]
\]
and
\[\mu_{st} = \E[X_s, X_t] = \prob(X_s = 1, X_t = 1) \in [0,1].
\]

There are many relationships we can find for the mean parameter, for exmaple
\[
\mu_{st} = \prob(X_s=1, X_t=1) \leq \max\{\prob(X_s=1), \prob(X_t=1)\} = \max\{\mu_s, \mu_t\}
\]
and many others.
(But the relationship we mentioned in the class, which is $\sum_t \mu_{st} = \mu_s$, doesn't make sense to me.
Any comments?) 

However, it is not easy to enumerate all constaints that could fully characterize $\Mc$. In fact, the polytope
\[
\Mc = \left\{ \{\mu_s\}_{s\in V} \; \{\mu_{st}\}_{(s,t) \in E} \; | \;
\mu_s = \E_\prob[X_s] \; \mu_{st} = \E_\prob[X_sX_t] \text{ for some distribution } \prob \right\}
\]
is called a cut/correlation polytope and it has exponentially many halfspaces. It is an important topic
in combinatorics and people don't fully understand it even now.

\subsection{Over Complete Family}
Suppose $X = (X_1, X_2, \ldots, X_p)$, $X_s \in \{0, 1, \ldots, r-1\}$. The sufficient statistics are
$\{\1_{s,j}(x)\}_{s\in V}$ and $\{\1_{st,jk}(x)\}_{(s,t) \in E}$, so the mean parameter is
\[
\mu_{s,j} = \E[\1_{s,j}(x)] = \prob(X_s = j)
\]
and
\[
\mu_{st,jk} = \E[\1_{st,jk}(x)] = \prob(X_s = j, X_t = k).
\]

Since the model is over complete, we have marginalization constraints such as
\[
\sum_k \mu_{st,jk} = \mu_{s,j} \text{ for all } s,t,j.
\]
Due to this reason (maybe?), the corresponding polytope $\Mc$ is called the margingal polytope, which,
as in the case of Ising model, could have exponential facets and is complicated in nature. The marginal
polytope is sometimes denoted as $\Mc(G)$, where $G=(V, E)$ is the underlining pairwise constaint.

\section{Mean Parameter in Inference}
Consider an exponential family
\[
p(x, \theta) = \exp\{\langle \theta, \phi(x) \rangle - A(\theta)\}.
\]
where $X = (X_1, \ldots, X_p)$ and $\{\phi_\alpha(x)\}_{\alpha \in I}$ are given. Suppose we have $n$
independent samples $x^{(1)}, \ldots, x^{(n)} \in \R^p$. To find $\theta$, the traditional way is to
maximize the log-likelihood
\[
\max_{\theta \in \Omega} \sum_{i=1}^n \log p_\theta(x^{(i)}).
\]

It turns out that for exponential family, it has nice form (since $\log$ of $\exp$ is just linear)
\[
\log p_\theta(x^{(i)}) = \langle \theta, \phi(x^{(i)}) \rangle - A(\theta),
\]
\[
\frac{1}{n} \sum_{i=1}^n \log p_\theta(x^{(i)})
= \langle \theta, \frac{1}{n} \sum_{i=1}^n \phi(x^{(i)}) \rangle - A(\theta).
\]
Let $\hat{\mu} = \frac{1}{n} \sum_{i=1}^n \phi(x^{(i)})$, then the maximization could be written as
\[
\max_{\theta \in \Omega} \langle \theta, \hat{\mu} \rangle - A(\theta).
\]
Notice that $\hat{\mu}$ is convex combination of $\phi(x^{(i)})$, so $\hat{\mu} \in \Mc$.

\section{Properties of $A(\theta)$}
\begin{enumerate}[(a)]
\item \label{convexity}
$A(\theta)$ is a convex function.
\item \label{strong_convexity}
$A(\theta)$ is strongly convex if the exponential family is minimal.
\item \label{derivatives}
$\nabla A(\theta) = \E[\phi(x)]$, $\nabla^2 A(\theta) = \mathrm{Cov} (\phi(x))$.
\end{enumerate}
We have already proved (\ref{derivatives}) in previous class. (\ref{convexity}) and (\ref{strong_convexity})
are important since it guarantees the convexity of log-likelihood. To prove (\ref{convexity}) assuming
it is twice-differentiable, just notice $\nabla^2 A(\theta) = \mathrm{Cov} [\phi(x)] \succeq 0$.
\begin{proof}[Proof of (\ref{strong_convexity})]
If the exponential family is minimal, then $\langle a, \phi(x) \rangle$ is not constant for all $a$. If $X$
is discrete, then this implies that $\mathrm{Var} (\langle a, \phi(x) \rangle) > 0$. If $X$ is continuous and
$\phi(x)$ is continuous, then it also implies that $\mathrm{Var} (\langle a, \phi(x) \rangle) > 0$. The last
step is to notice that $a^T \nabla^2 A(\theta) a = \mathrm{Var}(\langle a, \phi(x) \rangle)$.
\end{proof}

\end{document}






