\documentclass[11pt]{article}

\usepackage{amsmath}
\usepackage{amssymb}
\usepackage{amsthm}
\usepackage[retainorgcmds]{IEEEtrantools}
\usepackage{mathrsfs}
\usepackage{bbm}
\usepackage{enumerate}
\usepackage{hyperref}

% Abbreviations
\newcommand{\ie}{\textit{i.e.}}
\newcommand{\eg}{\textit{e.g.}}

% amsthm stuff:  Theorem, Definition, Lemma, Corollary
\theoremstyle{definition} \newtheorem{definition}{Definition}
\theoremstyle{definition} \newtheorem{observation}[definition]{Observation}
\theoremstyle{definition} \newtheorem{proposition}[definition]{Proposition}
\theoremstyle{definition} \newtheorem{lemma}[definition]{Lemma}
\theoremstyle{definition} \newtheorem{corollary}[definition]{Corollary}
\theoremstyle{definition} \newtheorem{theorem}[definition]{Theorem}

%Symbols
\newcommand{\R}{\mathbb{R}}
\newcommand{\M}{\mathbb{M}}
\newcommand{\LB}{\mathbb{L}}
\newcommand{\XS}{\mathcal{X}}
\newcommand{\Mu}{\mathcal{M}}
\newcommand{\Prob}{\mathbb{P}}
\newcommand{\E}{\mathbb{E}}
\newcommand{\Ind}{\mathbbm{1}}

% Math
\newcommand{\floor}[1]{\left\lfloor #1 \right\rfloor}
\newcommand{\ceil}[1]{\left\lceil #1 \right\rceil}
\newcommand{\size}[1]{\left\arrowvert #1 \right\arrowvert}
\newcommand{\abs}[1]{\left\arrowvert #1 \right\arrowvert}
\newcommand{\Oh}[1]{{\mathcal O}\left({#1}\right)}
\newcommand{\oh}[1]{{o}\left({#1}\right)}
\newcommand{\Om}[1]{{\Omega}\left({#1}\right)}
\newcommand{\om}[1]{{\omega}\left({#1}\right)}
\newcommand{\Th}[1]{{\Theta}\left({#1}\right)}
\newcommand{\cbr}[1]{\left\{ #1 \right\}}
\newcommand{\sbr}[1]{\left[ #1 \right]}
\newcommand{\seq}[1]{\langle #1 \rangle}

\title{CS395T Graphical Models: Scribe Notes}
\author{Nevzat Onur Domani\c{c} \\ onur@cs.utexas.edu}
\date{October 29, 2012}
\begin{document}
\maketitle

\section{Properties of $A(\theta)$}
\label{sec:properties-a}

Consider the distribution $p_\theta$ for some $\theta \in \Omega$.
Remember $\mu = \E_{p_\theta}\sbr{\phi(X)} = \nabla A(\theta)$.

\textbf{I.} $\nabla A:\Omega \to \Mu$ is one-to-one iff the exponential family
is minimal.

\begin{proof}
  If the exponential family is not minimal then $\exists \theta_1,
  \theta_2$ such that $p_{\theta_1} = p_{\theta_2}$ and $\theta_1
  \neq \theta_2$.  Then $\nabla A(\theta_1) = \nabla A(\theta_2)$
  hence $\nabla A$ is not one-to-one.

  If the exponential family is minimal then $A$ is strictly convex,
  $\nabla^2 A(\theta) \succ 0$, so $\nabla A$ is one-to-one.
\end{proof}

\textbf{II.} $\forall \mu \in Int(\Mu)$, $\exists \theta(\mu) \in
\Omega$ such that $\E_{p_{\theta(\mu)}}\sbr{\phi(x)} = \mu$, in other
words $\nabla A(\theta(\mu)) = \mu$, so $\nabla A$ is onto $Int(\Mu)$.

\section{Fenchel-Conjugate of $A$}
\label{sec:fenchel-conjugate-a}

For $\mu \in \R^d$ the Fenchel-Conjugate function of $A$ is defined as:
\[ A^*(\mu) := \sup_{\theta \in \Omega} \cbr{\seq{\theta, \mu} -
  A(\theta)} \]

\begin{theorem}
  \hfill
  \begin{enumerate}[(I)]
  \item $\forall \mu \in Int(\Mu)$, $A^*(\mu) = -H(p_{\theta(\mu)})$
    (negative entropy)

  \item $\forall \mu \notin Cl(\Mu)$, $A^*(\mu) = +\infty$

  \item $\forall \mu \in \partial \Mu$, there exists a sequence
    $\cbr{\mu_n} \subset Int(\Mu)$ converging to $m$, and $A^*(\mu) =
    \lim\limits_{n \to \infty} A^*(\mu_n)$
  \end{enumerate}

  \begin{proof}
    We prove property (I).  Since $\mu \in Int(\Mu)$, $\exists
    \theta(\mu) \in \Omega$ such that $\nabla A(\theta(\mu)) = \mu$.
    Remember the definition of $A^*$:
    \[ A^*(\mu) = \sup_{\theta \in \Omega} \cbr{\seq{\theta, \mu} -
      A(\theta)} \]
    The $\theta$ that achieves the supremum is $\theta(\mu)$, so
    $A^*(\mu) = \seq{\theta(\mu), \mu} - A(\theta(\mu))$.

    From the definition of the entropy:
    \begin{IEEEeqnarray*}{rCl}
      -H(p_\theta) &=& \E_{p_\theta}\sbr{\log p_\theta(x)} \\
      &=& \E_{p_\theta}\sbr{\seq{\theta, \phi(x)} - A(\theta)} \\
      &=& \seq{\theta, \mu(\theta)} - A(\theta)
    \end{IEEEeqnarray*}
    so $-H(p_{\theta(\mu)}) = \seq{\theta(\mu), \mu} - A(\theta(\mu))
    = A^*(\mu)$.
  \end{proof}
\end{theorem}

\section{Variational Principle}
\label{sec:vari-princ}

\begin{theorem}
  \[ A(\theta) = \sup_{\mu \in \Mu}\cbr{\seq{\theta, \mu} -
    A^*(\mu)} \]
  In other words, $A(\theta) = A^{**}(\theta)$, \ie, $A$ is the
  Fenchel-Conjugate of $A^*$.
\end{theorem}

Note that this is a convex optimization problem since both
$\seq{\theta, \mu}$ and the negative entropy is convex.  However
specifying the objective function and the convex set takes exponential
time.

So in Graphical Model Inference we
\begin{enumerate}[(a)]
\item Approximate $\Mu$

\item Approximate entropy $-A^*(\mu)$
\end{enumerate}

How to compute $A^*(\mu)$:
\begin{enumerate}[(i)]
\item Obtain $\theta(\mu)$ such that $\nabla A(\theta(\mu)) = \mu$.

\item Compute $-H(p_{\theta(\mu)})$
\end{enumerate}
Let's look at an example:

\subsection{Example: Bernoulli Distribution}
\label{sec:exampl-bern-distr}

Recall $p(x; \theta) = \exp \cbr{x \theta - \log(1 + \exp(\theta))}$
for $x \in \cbr{0, 1}$.

Let's check if $A^*(\mu) = \mu \log \mu + (1 - \mu)\log(1 - \mu)$.
From the definition of the Fenchel-Conjugate we have:
\[ A^*(\mu) = \sup_{\theta}\cbr{\theta \mu - \log(1 + \exp(\theta))} \]
Taking derivative w.r.t. $\theta$ and setting it to zero:
\[ \mu = \frac{\exp \theta}{1 + \exp \theta} \implies 
   \exp \theta = \frac{\mu}{1-\mu} \implies
   \theta = \log\frac{\mu}{1-\mu} \]
so
\begin{IEEEeqnarray*}{rCl}
  A^*(\mu) &=& \theta \mu - \log(1 + \exp(\theta(\mu))) \\
  &=& \mu \log \frac{\mu}{1-\mu} + \log(1 - \mu) \\
  &=& \mu \log \mu + (1 - \mu) \log(1 - \mu)
\end{IEEEeqnarray*}

Now let's verify $A(\theta)$.  We have:
\begin{equation}
  \label{eq:Ber-A}
  A(\theta) = \sup_\mu \cbr{\theta \mu - \mu \log \mu - (1 - \mu)
    \log(1 - \mu)}
\end{equation}
Taking derivative w.r.t. $\mu$ and setting it to zero:
\[ \theta = \log \mu +1-1 - \log(1-\mu) = \log\frac{\mu}{1-\mu}
   \implies \mu = \frac{\exp\theta}{1 + \exp\theta} \]
Substituting $\mu$ in equation~(\ref{eq:Ber-A}) gives $A(\theta) =
\log(1 + \exp\theta)$.

\subsubsection{Approximating Polytope in Discrete Models}
\label{sec:appr-polyt-discr}

$X = (X_1, \dotsc, X_p)$ where $X_s \in \XS = \cbr{0, \dotsc, r-1}$.
We have:
\[ p_\theta(X) = \exp \cbr{\sum_{s,j} \Ind_{s,j}(X)\theta_{sj} + 
   \sum_{\substack{(s,t) \in E\\ j,k}} \Ind_{st,jk}(X)\theta_{st,jk} -
   A(\theta)} \]

We introduce the following shorthand notations:
\begin{IEEEeqnarray*}{rCl}
  \theta_s(X_s) &=& \sum_j \Ind_{s,j}(X_s) \theta_{s,j} \\
  \theta_{st}(X_s, X_t) &=& \sum_{j,k} \Ind_{st,jk}(X_s, X_t) \theta_{st,jk}
\end{IEEEeqnarray*}
so $p_\theta(X)$ becomes $=\exp\cbr{\sum\limits_s \theta_s(X_s) +
  \sum\limits_{(s,t) \in E} \theta_{st}(X_s, X_t) - A(\theta)}$.

Let's have
\begin{IEEEeqnarray*}{rCl}
  \phi_s(X_s) &=& \sum_{j \in \XS} \Ind(X_s = j)\phi_s(j) \\
  \text{and} && \\
  \phi_{st}(X_s, X_t) &=& \sum_{j,k \in \XS} \Ind(X_s=j, X_t=k)\phi_{st}(j,k)
\end{IEEEeqnarray*}
so any sufficient statistics over discrete random variables can be
expressed as a linear combination of indicator functions, thus the
overcomplete representation is general.

Now let's look at the mean parameters.  Remember that the Marginal Polytope
$\M(G)$ is defined as:
\[ \M(G) = \cbr{\mu : \exists p, \mu = \E_p\sbr{\phi(x)}}\]

The mean parameters are:
\begin{IEEEeqnarray*}{rCl}
  \mu_{s,j} &=& \E\sbr{\Ind_{s,j}(X_s)} = \Prob(X_s = j) \\
  \mu_{st,jk} &=& \E\sbr{\Ind_{st,jk}(X_s, X_t)} = \Prob(X_s=j, X_t=k)
\end{IEEEeqnarray*}
So we define:
\begin{IEEEeqnarray*}{rCl}
  \mu_{s}(X_s) &=& \sum_{j} \Ind_{s,j}(X_s) \mu_{s,j} \\
  \mu_{st}(X_s,X_t) &=& \sum_{jk} \Ind_{st,jk}(X_s,X_t) \mu_{st,jk}
\end{IEEEeqnarray*}
Then the Marginal Polytope is:
\[ \M(G) = \cbr{\mu_{s}(X_s), \mu_{st}(X_s,X_t) : \parbox{5.5cm}{node-wise and
    pair-wise \\ marginals of some distribution $p$}} \]

Note that we have the following set of constraints:
\begin{itemize}
\item 
  \begin{IEEEeqnarray*}{rl}
    &\mu_{s,j} \in [0, 1] \\
    &\mu_{s}(X_s) \in [0, 1] \\
    &\mu_{st}(X_s, X_t) \in [0, 1]
  \end{IEEEeqnarray*}

\item Normalization Constraints
  \begin{IEEEeqnarray*}{rl}
    \sum_{X_s} &\mu_{s}(X_s) = 1 \\
    \sum_{X_s,X_t} &\mu_{st}(X_s, X_t) = 1
  \end{IEEEeqnarray*}

\item Marginalization Constraints
  \begin{IEEEeqnarray*}{rl}
    \sum_{X_t} &\mu_{st}(X_s, X_t) = \mu_{s}(X_s) \\
    \sum_{X_s} &\mu_{st}(X_s, X_t) = \mu_{t}(X_t)
  \end{IEEEeqnarray*}
\end{itemize}

These are called the ``local'' constraints and the number
of constraints is $\Oh{\size{V} + \size{E}}$.

We define:
\[ \LB(G) = \cbr{
  \begin{array}{l}
    T_s(X_s) \\
    T_{st}(X_s, X_t)
  \end{array}
  : \text{$T_{s}, T_{st}$ satisfy ``local'' constraints}
} \]
where $T_{s}$ and $T_{st}$ are called the pseudo-marginals.

\end{document}

%  LocalWords:  iff Fenchel supremum overcomplete polytope
